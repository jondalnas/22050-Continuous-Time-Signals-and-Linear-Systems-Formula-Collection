Fourier transformationen eksistere kun for konvergerende funktioner
\subsection{Fourier transformation ligningen}
	\begin{align}
		X(\omega)=\int_{-\infty}^{\infty}x(t)e^{-j\omega t}\,dt
	\end{align}

\subsection{Omvendt Fourier transformation}
	\begin{align}
		x(t)=\frac{1}{2\pi}\int_{-\infty}^{\infty}S(\omega)e^{j\omega t}\,d\omega
	\end{align}

\subsection{Spektre}
	\begin{itemize}
		\item Amplitudespektrum: $|X(\omega)|$
		\item Fasespektrum: $\angle X(\omega)$
	\end{itemize}

\subsection{Brugbare funktioner}
	\begin{itemize}
		\item $F\{\delta(t)\}=1$
		\item $F^{-1}\{\delta(\omega)\}=\frac{1}{2\pi}$
		\item $F^{-1}\{2\pi\delta(\omega)\}=1$
		\item $F^{-1}\{2\pi\delta(\omega-\omega_0)\}=e^{j\omega_0t}$
		\item $F^{-1}\{\pi(\delta(\omega-\omega_0)+\delta(\omega+\omega_0))\}=\cos(\omega_0t)$
		\item $F\{\text{sgn}(t)\}=\frac{2}{j\omega}$
		\item $F\{u(t)\}=\pi\delta(\omega)+\frac{1}{j\omega}$
	\end{itemize}

\subsection{Fourier serie}
	\subsubsection{Tidsskalering}
		\begin{align}
			x(at)\leftrightarrow\frac{1}{|a|}X(\frac{\omega}{a})
		\end{align}

	\subsubsection{Tidsforskydning}
		\begin{align}
			x(t-t_0)\leftrightarrow X(\omega)e^{-j\omega t_0}
		\end{align}

	\subsubsection{Frekvensforskydning}
		\begin{align}
			x(t)e^{j\omega_0t}\leftrightarrow X(\omega-\omega_0)
		\end{align}

	\subsubsection{Modulering}
		\begin{align}
			x(t)\cos(\omega_0t)\leftrightarrow\frac{1}{2}X(\omega-\omega_0)+\frac{1}{2}X(\omega+\omega_0)
		\end{align}

\subsection{Foldning i tid og frekvensdomænet}
	\begin{align}
		h(t)*x(t)\leftrightarrow H(\omega)X(\omega)
	\end{align}

	\begin{align}
		H(\omega)*X(\omega)\leftrightarrow h(t)x(t)
	\end{align}

	\subsubsection{Tidsafhængig integration}
		\begin{align}
			\int_{-\infty}^{t}=x(t)*u(t)\leftrightarrow\pi X(0)\delta(\omega)+\frac{X(\omega)}{j\omega}
		\end{align}
	
	\subsubsection{Tidsafhængig differentiation}
		\begin{align}
			\frac{dx}{dt}\leftrightarrow j\omega X(\omega)
		\end{align}

\subsection{Sampeling}
	\subsubsection{Impulstog}
		\begin{align}
			\delta_s(t)=\sum_{n=-\infty}^{\infty}\delta(t-n\Delta t)
		\end{align}

		\begin{align}
			\Delta_s(\omega)=\sum_{n=-\infty}^{\infty}\frac{\omega_s}{2\pi}\delta(\omega-n\omega_s)
		\end{align}

	\subsubsection{Nyquist sætning}
		\begin{align}
			x_s(t)=x(t)*\delta_s(t)\leftrightarrow X_s(\omega)=\frac{\omega_s}{(2\pi)^2}\sum_{n=-\infty}^{\infty}X(\omega-n\omega_s)
		\end{align}

		If the sampling frequency is too small, adjacent extensions of the frequency spectrum overlap and irreversible aliasing occurs ($\omega_s>2B$, hvor $B$ er båndbreden)
	
\subsection{Parseval's sætning}
	Energien i et signal er arealet under $|X(\omega)|^2$
	\begin{align}
		E_f=\frac{1}{2\pi}\int_{-\infty}^{\infty}|X(\omega)|^2\,d\omega
	\end{align}

\subsection{Fourier transformation spektrums tabel}
	\begin{tabular}{|c|c|c|}
		\hline
		&Aperiodisk i tid&Periodisk i tid\\
		\hline
		Kontinuær i tid&Aperiodisk og kontinuær frekvensspektrum&Aperiodisk og diskret frekvensspektrum\\
		\hline
		diskret i tid&Periodisk og kontinuær frekvensspektrum&Periodisk og diskret frekvensspektrum\\
		\hline
	\end{tabular}

\subsection{Filtre}
	\subsubsection{Aplitude og fase}
		\begin{align}
			H(\omega)=\frac{P(\omega)}{Q(\omega)}
		\end{align}
		\begin{align}
			|H(\omega)|=\frac{|P(\omega)|}{|Q(\omega)|}
		\end{align}
		\begin{align}
			\angle H(\omega)=\angle P(\omega)-\angle Q(\omega)
		\end{align}

	\subsubsection{Cut-off frekvens}
		Frekvensen hvor halvdelen af effekten af signalet er forsvundet
		\begin{align}
			|H(\omega)|=-3\text{dB}
		\end{align}
	
	\subsubsection{Frekvens- og impedansskalering}
		Ved at multiplicere vinkelhastigheden med en konstant og derefter dividere komponentværdierne med den samme værdi, kan vi skalere krekvensen af et filter\\
		Ved at multiplicere og dividere komponentværdier med den samme konstant, kan vi opnå at skalere værdierne uden at skalere frekvensen